During the months of June and July I decided to do my internship at the Computer Laboratory of Cambridge with Thomas Gazagnaire (former student from the ENS of Lyon and Rennes). I chose to do my internship with Thomas Gazagnaire and his team as much for the programming language they work with (OCaml) as for the projects of the team (MirageOS, Irmin, ...). 
\paragraph{} Thomas and I chose the subject of the internship before the beginning of it which allowed me to document myself on the field I was going to work on. We chose the aim of the internship at my arrival and therefore decided that I was going to try to adapt existing algorithms (as described in \cite{DBLP:journals/cacm/Lamport78}) to other cases. However after some work we realized it was not going to work and therefore decided that we were going to try to establish a new algorithm using Bloom Filters, the idea of using Bloom Filters coming from the work we had already done, while trying to adapt algorithms. Once the algorithm established I had to implement, test it and shape it so that it could be used in a library, while documenting myself on the state of the art in terms of Bloom Filters and Hash function families (this part was made easier thanks to the help of Magnus Skjegstad, a colleague of Thomas). Once the code implemented we defined what were the invariants of the functions in the library we 
wrote.

\paragraph{} In conclusion we underlined the difficulties of adapting the work already done in the cases where there are a finite number of processes and a unique biggest common ancestor. We wrote a library\footnote{https://github.com/samoht/ocaml-bloom-filters} of functions enabling a user to synchronize persistent DAGs, with a low cost in term of data exchanged and of complexity. We dealt with the side cases due to false positive in Bloom Filters and tried to find a "good" hash family function that ensures the user that the false positive rate will be low.