This report gives the headline and the results of my two months of internship with the NetOS\footnote{http://www.cl.cam.ac.uk/research/srg/netos/} (Networks and Operating Systems group) group in the Computer Lab in Cambridge. My work has been supervised by Thomas Gazagnaire. The NetOS group is included in the System Research Group of the Computer Lab. One of the project of the NetOS team is MirageOS\footnote{http://www.openmirage.org/}, which is a library operating system that constructs unikernels for secure, high-performance network applications across a variety of cloud computing and mobile platforms. As a part of the MirageOS project, I worked on Irmin, a Git-like distributed, branchable storage.

\paragraph{} The Irmin storage enables the user to synchronise distributed data structures. Those data structures are persistent graphs that are Merkle DAGs (see \cite{Becker_merklesignature}) which are used by many systems (such as Git, Ori (see \cite{Mashtizadeh:2013:RHG:2517349.2522721}) , etc ...). My contribution to the project was the elaboration of an efficient algorithm that enables users to synchronise their history in an asymetric way (meaning that a client pulls data from a server) which is to compute the difference $G \setminus G'$ for 2 graphs described as before. We did not make any assumption on the shape of the DAGS, meaning that the detailed algorithm can be used for any number of processes working simultaneously.

\paragraph{} In order to underline the results of the intership as well as the research that has been done, this report starts with a presentation on the state of the art in terms of DAGs synchronization (Vector Clock) followed by a presentation of Bloom filters (as they are used in the algorithm). This introduction is followed by a presentation of the results of the internship : some preliminary results, the main algorithm (which signature can be found in the Appendix), some considerations on the hash family that are used for the Bloom Filters and an evaluation of the algorithm.

% My internship began with a bibliographic work (that can be found in the bibliography 4.1) during which I had to understand the subject. An algorithm to synchronize persistant DAGs already exists and is implemented in the Git library, however my work was to find the limitations of this existing algorithm and to use the results that are known in particular cases (fixed number of processes, etc ...) in order to design an algorithm that could be used in the Irmin database. Once the algorithm designed I implemented it and tested if in OCaml.