Our algorithm uses Bloom filters, therefore some false positive membership query can happen, even if we chose the hash family so that the probibility that a false positive occur is small. In the algorithm, membership queries only occur when the server is computing the difference between its history and the client one using Bloom filter. Let us consider that the server just received a Bloom filter and a list of "interesting nodes" from which it shall start exploring its ancestors, two different false positive cases can occur :
\begin{enumerate}
 \item A false positive occurs in a Bloom filter, therefore the server will stop exploring up at this node and will add all of its sons to the ring to send to the client. However the client will receive the ring and some node will not have any fathers in its history, those nodes shall be the one in the "interesting nodes" list otherwise it means that it is a node the server considered in the history of the client and therefore can detect that a false positive occured. In such case the client can send back the same Bloom filter while signaling to the server which nodes were false-positive, hence the server will go accross the false-positive node and continue the exploration with the correct Bloom filter, for an example see \ref{bffp}.
 \item A false positive occurs in a border, there is no way to discover such a false positive until the end of the algorithm, when the client has sent all of its Bloom filters. At this point if the algorithm occured without any false positive, there should be no more "interesting nodes", however in the case of a false positive in a border, there remains some "interesting nodes". Considering that the border can not be trusted to test the membership, the client sends its border to the server that tests the membership to the border for every ancestors of the remaining "interesting nodes", and send the result back to the client who verifies that the received nodes are in its history. If a node is not in its history, then it raised a false positive in the border, once we know all the false positive nodes, we send them to the server that will ignore them while starting over all of the exploration. We can remark that if a node was in the history of the client but was raising a false positive in the border, then we 
do not have a problem because the node was already in the client history, as were all of its ancestors, therefore we do not need this node to be further explored on the server side. Having a false positive in a border is a very rare case because the border are rather small.
\end{enumerate}
\begin{figure}[H]
 \centering
 \resizebox{\textwidth}{!}{
 \begin{tabular}{c|c|c|c|c}
 \begin{subfigure}[b]{0.2\textwidth}
  \centering
  \begin{tikzpicture}[->,>=stealth',shorten >=1pt,auto,node distance=3cm,
  thick,main node/.style={circle,fill=blue!20,draw,font=\sffamily\Large\bfseries}]
  \foreach \place/\x in {{(0,0)/1}, {(0,1)/2}}
  \node[arn_n] (a\x) at \place {\x};
  
  
  \node[arn_n] (a5) at (1,4.5) {5};
  \node[arn_n] (a13) at (2,5) {13};
  \node[arn_n] (a14) at (2,6) {14};
  
  
  \node[arn_n] (a10) at (2,3) {10};
  \node[arn_n] (a7) at (1.5,2) {7};
  \node[arn_n] (a3) at (0,2) {3};
  
  \node[arn_n] (a9) at (1,3) {9};
  \node[arn_n] (a12) at (2,4) {12};
  \node[arn_n] (a4) at (0,3) {4};
  %   Alice history
  \path[thin] (a14) edge (a5);
  \path[thin] (a5) edge (a4);
  \path[thin] (a4) edge (a3);
  \path[thin] (a3) edge (a2);
  \path[thin] (a2) edge (a1);
  \path[thin] (a9) edge (a4);
  \path[thin] (a7) edge (a2);
%   both history
  \path[thin] (a14) edge (a13);
  \path[thin] (a13) edge (a12);
  \path[thin] (a12) edge (a9);
  \path[thin] (a12) edge (a10);
  
  \path[thin] (a9) edge (a7);
  \path[thin] (a10) edge (a7);
  \end{tikzpicture}
  \caption{Alice's ancestors}

  \end{subfigure}
  &
  \begin{subfigure}[b]{0.2\textwidth}
\centering
 \begin{tikzpicture}[->,>=stealth',shorten >=1pt,auto,node distance=3cm,
  thick,main node/.style={circle,fill=blue!20,draw,font=\sffamily\Large\bfseries}]
  
  
  
  \node[arn_r] (a1) at (0,0) {1};
  \node[arn_r] (a2) at (0,1) {2};
  
  \node[arn_y] (a5) at (1,4.5) {5};
  \node[arn_n] (a13) at (2,5) {13};
  \node[arn_n] (a14) at (2,6) {14};
  
  
  \node[arn_n] (a10) at (2,3) {10};
  \node[arn_y] (a7) at (1.5,2) {7};
  \node[arn_r] (a3) at (0,2) {3};
  
  \node[arn_rs] (a9) at (1,3) {9};
  \node[arn_n] (a12) at (2,4) {12};
  \node[arn_r] (a4) at (0,3) {4};
  
  %   Alice history
  \path[thin] (a14) edge (a5);
  \path[thin] (a5) edge (a4);
  \path[thin] (a4) edge (a3);
  \path[thin] (a3) edge (a2);
  \path[thin] (a2) edge (a1);
  \path[thin] (a9) edge (a4);
  \path[thin] (a7) edge (a2);
%   both history
  \path[thin] (a14) edge (a13);
  \path[thin] (a13) edge (a12);
  \path[thin] (a12) edge (a9);
  \path[thin] (a12) edge (a10);
  
  \path[thin] (a9) edge (a7);
  \path[thin] (a10) edge (a7);
  \end{tikzpicture}
  \caption{Alice explores and 5 is a false positive}
\end{subfigure}
  &
  \begin{subfigure}[b]{0.2\textwidth}
 \centering
 \begin{tikzpicture}[->,>=stealth',shorten >=1pt,auto,node distance=3cm,
  thick,main node/.style={circle,fill=blue!20,draw,font=\sffamily\Large\bfseries}]
  
  

  \node[arn_rb] (emph) at (1,3) {};
  \node[arn_b] (a1) at (0,0) {1};
  \node[arn_b] (a2) at (0,1) {2};
  
  \node[arn_b] (a5) at (1,4.5) {5};
  \node[arn_n] (a13) at (2,5) {13};
  \node[arn_n] (a14) at (2,6) {14};
  
  
  \node[arn_n] (a10) at (2,3) {10};
  \node[arn_b] (a7) at (1.5,2) {7};
  \node[arn_b] (a3) at (0,2) {3};
  
  \node[arn_b] (a9) at (1,3) {9};
  \node[arn_n] (a12) at (2,4) {12};
  \node[arn_b] (a4) at (0,3) {4};
%   \node[cloud, fill=gray!20, cloud puffs=16, cloud puff arc= 100,minimum width=2em, minimum height=2em, aspect=1] (cloud) at (5,2) {};

  %   Alice history
  \path[thin] (a14) edge (a5);
  \path[thin] (a5) edge (a4);
  \path[thin] (a4) edge (a3);
  \path[thin] (a3) edge (a2);
  \path[thin] (a2) edge (a1);
  \path[thin] (a9) edge (a4);
  \path[thin] (a7) edge (a2);
%   both history
  \path[thin] (a14) edge (a13);
  \path[thin] (a13) edge (a12);
  \path[thin] (a12) edge (a9);
  \path[thin] (a12) edge (a10);
  
  \path[thin] (a9) edge (a7);
  \path[thin] (a10) edge (a7);
  \end{tikzpicture}
  \caption{Alice deals with border and Bloom Filter nodes}
\end{subfigure}
&

  \begin{subfigure}[b]{0.2\textwidth}
 \begin{tikzpicture}[->,>=stealth',shorten >=1pt,auto,node distance=3cm,
  thick,main node/.style={circle,fill=blue!20,draw,font=\sffamily\Large\bfseries}]
  
  
  
  \node[arn_n] (a1) at (0,0) {1};
  \node[arn_n] (a2) at (0,1) {2};
  
  \node[arn_r] (a5) at (1,4.5) {5};
  \node[arn_n] (a13) at (2,5) {13};
  \node[arn_rs] (a14) at (2,6) {14};
  
  
  \node[arn_n] (a10) at (2,3) {10};
  \node[arn_n] (a7) at (1.5,2) {7};
  \node[arn_n] (a3) at (0,2) {3};
  
  \node[arn_n] (a9) at (1,3) {9};
  \node[arn_n] (a12) at (2,4) {12};
  \node[arn_n] (a4) at (0,3) {4};
  
  %   Alice history
  \path[thin] (a14) edge (a5);
  \path[thin] (a5) edge (a4);
  \path[thin] (a4) edge (a3);
  \path[thin] (a3) edge (a2);
  \path[thin] (a2) edge (a1);
  \path[thin] (a9) edge (a4);
  \path[thin] (a7) edge (a2);
%   both history
  \path[thin] (a14) edge (a13);
  \path[thin] (a13) edge (a12);
  \path[thin] (a12) edge (a9);
  \path[thin] (a12) edge (a10);
  
  \path[thin] (a9) edge (a7);
  \path[thin] (a10) edge (a7);
  \end{tikzpicture}
  \caption{Alice explores again starting at 5}
\end{subfigure}
&
  \begin{subfigure}[b]{0.2\textwidth}
 \begin{tikzpicture}[->,>=stealth',shorten >=1pt,auto,node distance=3cm,
  thick,main node/.style={circle,fill=blue!20,draw,font=\sffamily\Large\bfseries}]
  
  
   \node[arn_rb] (emph) at (2,6) {};
  \node[arn_n] (a1) at (0,0) {1};
  \node[arn_n] (a2) at (0,1) {2};
  
  \node[arn_b] (a5) at (1,4.5) {5};
  \node[arn_b] (a13) at (2,5) {13};
  \node[arn_n] (a14) at (2,6) {14};
  
  
  \node[arn_n] (a10) at (2,3) {10};
  \node[arn_n] (a7) at (1.5,2) {7};
  \node[arn_n] (a3) at (0,2) {3};
  
  \node[arn_n] (a9) at (1,3) {9};
  \node[arn_n] (a12) at (2,4) {12};
  \node[arn_n] (a4) at (0,3) {4};
  
  %   Alice history
  \path[thin] (a14) edge (a5);
  \path[thin] (a5) edge (a4);
  \path[thin] (a4) edge (a3);
  \path[thin] (a3) edge (a2);
  \path[thin] (a2) edge (a1);
  \path[thin] (a9) edge (a4);
  \path[thin] (a7) edge (a2);
%   both history
  \path[thin] (a14) edge (a13);
  \path[thin] (a13) edge (a12);
  \path[thin] (a12) edge (a9);
  \path[thin] (a12) edge (a10);
  
  \path[thin] (a9) edge (a7);
  \path[thin] (a10) edge (a7);
  \end{tikzpicture}
  \caption{Alice deals with Bloom filter and border nodes}
\end{subfigure}
  \\
    \begin{subfigure}[b]{0.2\textwidth}
\centering
  \begin{tikzpicture}[->,>=stealth',shorten >=1pt,auto,node distance=3cm,
  thick,main node/.style={circle,fill=blue!20,draw,font=\sffamily\Large\bfseries}]
%   
  \node[arn_y] (a6) at (2,1) {6};
  \node[arn_y] (a7) at (1.5,2) {7};
  \node[arn_y] (a8) at (2.5,2) {8};
  \node[arn_ts] (a9) at (1,3) {9};
  \node[arn_ts] (a10) at (2,3) {10};
  \node[arn_ts] (a11) at (3,3) {11};
  \node[arn_pus] (a12) at (2,4) {12};
  \node[arn_pu] (a13) at (2,5) {13};
  \node[arn_pus] (a14) at (2,6) {14};
  
%   both history
  \path[thin] (a14) edge (a13);
  \path[thin] (a13) edge (a12);
  \path[thin] (a12) edge (a9);
  \path[thin] (a12) edge (a10);
  
  \path[thin] (a9) edge (a7);
  \path[thin] (a10) edge (a7);
%   Bob history
  \path[thin] (a12) edge (a11);
  \path[thin] (a11) edge (a8);
  \path[thin] (a10) edge (a8);
  \path[thin] (a7) edge (a6);
  \path[thin] (a8) edge (a6);
  \end{tikzpicture}
  \caption{Bob's ancestors}
  \end{subfigure}
  &
  $\cdots$
  &
  \begin{subfigure}[b]{0.2\textwidth}
  \centering
  \begin{tikzpicture}[->,>=stealth',shorten >=1pt,auto,node distance=3cm,
  thick,main node/.style={circle,fill=blue!20,draw,font=\sffamily\Large\bfseries}]
  
  \foreach \place/\x in {{(0,0)/1}, {(0,1)/2},{(0,2)/3},{(0,3)/4}, {(1,4.5)/5}}
  \node[arn_n] (a\x) at \place {\x};
%   Alice history
%   \path[thin] (a14) edge (a5);
  \node[arn_y] (a6) at (2,1) {6};
  \node[arn_y] (a7) at (1.5,2) {7};
  \node[arn_y] (a8) at (2.5,2) {8};
  \node[arn_ts] (a9) at (1,3) {9};
  \node[arn_ts] (a10) at (2,3) {10};
  \node[arn_ts] (a11) at (3,3) {11};
  \node[arn_pus] (a12) at (2,4) {12};
  \node[arn_pu] (a13) at (2,5) {13};
  \node[arn_pus] (a14) at (2,6) {14};
  
  \path[thin] (a5) edge (a4);
  \path[thin] (a4) edge (a3);
  \path[thin] (a3) edge (a2);
  \path[thin] (a2) edge (a1);
  \path[thin] (a9) edge (a4);
  \path[thin] (a7) edge (a2);
%   both history
  \path[thin] (a14) edge (a13);
  \path[thin] (a13) edge (a12);
%   \path[thin] (a12) edge (a9);
  \path[thin] (a12) edge (a10);
  
  \path[thin] (a9) edge (a7);
  \path[thin] (a10) edge (a7);
%   Bob history
  \path[thin] (a12) edge (a11);
  \path[thin] (a11) edge (a8);
  \path[thin] (a10) edge (a8);
  \path[thin] (a7) edge (a6);
  \path[thin] (a8) edge (a6);
  \end{tikzpicture}
  \caption{Bob notices 5 should have a parent}
  \end{subfigure}
  &
  $\cdots$
  &
  \begin{subfigure}[b]{0.2\textwidth}
  \centering
  \begin{tikzpicture}[->,>=stealth',shorten >=1pt,auto,node distance=3cm,
  thick,main node/.style={circle,fill=blue!20,draw,font=\sffamily\Large\bfseries}]
  
  \foreach \place/\x in {{(0,0)/1}, {(0,1)/2},{(0,2)/3},{(0,3)/4}, {(1,4.5)/5}}
  \node[arn_n] (a\x) at \place {\x};
%   Alice history
%   \path[thin] (a14) edge (a5);
  \node[arn_y] (a6) at (2,1) {6};
  \node[arn_y] (a7) at (1.5,2) {7};
  \node[arn_y] (a8) at (2.5,2) {8};
  \node[arn_ts] (a9) at (1,3) {9};
  \node[arn_ts] (a10) at (2,3) {10};
  \node[arn_ts] (a11) at (3,3) {11};
  \node[arn_pus] (a12) at (2,4) {12};
  \node[arn_pu] (a13) at (2,5) {13};
  \node[arn_pus] (a14) at (2,6) {14};
  
  \path[thin] (a5) edge (a4);
  \path[thin] (a4) edge (a3);
  \path[thin] (a3) edge (a2);
  \path[thin] (a2) edge (a1);
  \path[thin] (a9) edge (a4);
  \path[thin] (a7) edge (a2);
%   both history
  \path[thin] (a14) edge (a5);
  \path[thin] (a14) edge (a13);
  \path[thin] (a13) edge (a12);
%   \path[thin] (a12) edge (a9);
  \path[thin] (a12) edge (a10);
  
  \path[thin] (a9) edge (a7);
  \path[thin] (a10) edge (a7);
%   Bob history
  \path[thin] (a12) edge (a11);
  \path[thin] (a11) edge (a8);
  \path[thin] (a10) edge (a8);
  \path[thin] (a7) edge (a6);
  \path[thin] (a8) edge (a6);
  \end{tikzpicture}
  \caption{Bob notices 5 should have a parent}
  \end{subfigure}
  \end{tabular}
}
\caption{How a false positive in a Bloom filter is detected and dealt with} \label{bffp}
\end{figure}

% \begin{figure}[H]
%  \centering
%   \begin{subfigure}[b]{0.2\textwidth}
%   \centering
%   \begin{tikzpicture}[->,>=stealth',shorten >=1pt,auto,node distance=3cm,
%   thick,main node/.style={circle,fill=blue!20,draw,font=\sffamily\Large\bfseries}]
%   \foreach \place/\x in {{(0,0)/1}, {(0,1)/2}}
%   \node[arn_n] (a\x) at \place {\x};
%   
%   
%   \node[arn_n] (a5) at (1,4.5) {5};
%   \node[arn_n] (a13) at (2,5) {13};
%   \node[arn_n] (a14) at (2,6) {14};
%   
%   
%   \node[arn_n] (a10) at (2,3) {10};
%   \node[arn_n] (a7) at (1.5,2) {7};
%   \node[arn_n] (a3) at (0,2) {3};
%   
%   \node[arn_n] (a9) at (1,3) {9};
%   \node[arn_n] (a12) at (2,4) {12};
%   \node[arn_n] (a4) at (0,3) {4};
%   %   Alice history
%   \path[thin] (a14) edge (a5);
%   \path[thin] (a5) edge (a4);
%   \path[thin] (a4) edge (a3);
%   \path[thin] (a3) edge (a2);
%   \path[thin] (a2) edge (a1);
%   \path[thin] (a9) edge (a4);
%   \path[thin] (a7) edge (a2);
% %   both history
%   \path[thin] (a14) edge (a13);
%   \path[thin] (a13) edge (a12);
%   \path[thin] (a12) edge (a9);
%   \path[thin] (a12) edge (a10);
%   
%   \path[thin] (a9) edge (a7);
%   \path[thin] (a10) edge (a7);
%   \end{tikzpicture}
%   \caption{Alice's ancestors}
% 
%   \end{subfigure}
%   \begin{subfigure}[b]{0.2\textwidth}
% \centering
%   \begin{tikzpicture}[->,>=stealth',shorten >=1pt,auto,node distance=3cm,
%   thick,main node/.style={circle,fill=blue!20,draw,font=\sffamily\Large\bfseries}]
% %   
%   \node[arn_y] (a6) at (2,1) {6};
%   \node[arn_y] (a7) at (1.5,2) {7};
%   \node[arn_y] (a8) at (2.5,2) {8};
%   \node[arn_ts] (a9) at (1,3) {9};
%   \node[arn_ts] (a10) at (2,3) {10};
%   \node[arn_ts] (a11) at (3,3) {11};
%   \node[arn_pus] (a12) at (2,4) {12};
%   \node[arn_pu] (a13) at (2,5) {13};
%   \node[arn_pus] (a14) at (2,6) {14};
%   
% %   both history
%   \path[thin] (a14) edge (a13);
%   \path[thin] (a13) edge (a12);
%   \path[thin] (a12) edge (a9);
%   \path[thin] (a12) edge (a10);
%   
%   \path[thin] (a9) edge (a7);
%   \path[thin] (a10) edge (a7);
% %   Bob history
%   \path[thin] (a12) edge (a11);
%   \path[thin] (a11) edge (a8);
%   \path[thin] (a10) edge (a8);
%   \path[thin] (a7) edge (a6);
%   \path[thin] (a8) edge (a6);
%   \end{tikzpicture}
%   \caption{Bob's ancestors}
%   \end{subfigure}
% \begin{subfigure}[b]{0.2\textwidth}
% \centering
%  \begin{tikzpicture}[->,>=stealth',shorten >=1pt,auto,node distance=3cm,
%   thick,main node/.style={circle,fill=blue!20,draw,font=\sffamily\Large\bfseries}]
%   
%   
%   
%   \node[arn_r] (a1) at (0,0) {1};
%   \node[arn_r] (a2) at (0,1) {2};
%   
%   \node[arn_y] (a5) at (1,4.5) {5};
%   \node[arn_n] (a13) at (2,5) {13};
%   \node[arn_n] (a14) at (2,6) {14};
%   
%   
%   \node[arn_n] (a10) at (2,3) {10};
%   \node[arn_y] (a7) at (1.5,2) {7};
%   \node[arn_r] (a3) at (0,2) {3};
%   
%   \node[arn_rs] (a9) at (1,3) {9};
%   \node[arn_n] (a12) at (2,4) {12};
%   \node[arn_r] (a4) at (0,3) {4};
%   
%   %   Alice history
%   \path[thin] (a14) edge (a5);
%   \path[thin] (a5) edge (a4);
%   \path[thin] (a4) edge (a3);
%   \path[thin] (a3) edge (a2);
%   \path[thin] (a2) edge (a1);
%   \path[thin] (a9) edge (a4);
%   \path[thin] (a7) edge (a2);
% %   both history
%   \path[thin] (a14) edge (a13);
%   \path[thin] (a13) edge (a12);
%   \path[thin] (a12) edge (a9);
%   \path[thin] (a12) edge (a10);
%   
%   \path[thin] (a9) edge (a7);
%   \path[thin] (a10) edge (a7);
%   \end{tikzpicture}
%   \caption{Alice explores and 5 is a false positive}
% \end{subfigure}
%  \begin{subfigure}[b]{0.2\textwidth}
%  \centering
%  \begin{tikzpicture}[->,>=stealth',shorten >=1pt,auto,node distance=3cm,
%   thick,main node/.style={circle,fill=blue!20,draw,font=\sffamily\Large\bfseries}]
%   
%   
% 
%   \node[arn_rb] (emph) at (1,3) {};
%   \node[arn_b] (a1) at (0,0) {1};
%   \node[arn_b] (a2) at (0,1) {2};
%   
%   \node[arn_b] (a5) at (1,4.5) {5};
%   \node[arn_n] (a13) at (2,5) {13};
%   \node[arn_n] (a14) at (2,6) {14};
%   
%   
%   \node[arn_n] (a10) at (2,3) {10};
%   \node[arn_b] (a7) at (1.5,2) {7};
%   \node[arn_b] (a3) at (0,2) {3};
%   
%   \node[arn_b] (a9) at (1,3) {9};
%   \node[arn_n] (a12) at (2,4) {12};
%   \node[arn_b] (a4) at (0,3) {4};
% %   \node[cloud, fill=gray!20, cloud puffs=16, cloud puff arc= 100,minimum width=2em, minimum height=2em, aspect=1] (cloud) at (5,2) {};
% 
%   %   Alice history
%   \path[thin] (a14) edge (a5);
%   \path[thin] (a5) edge (a4);
%   \path[thin] (a4) edge (a3);
%   \path[thin] (a3) edge (a2);
%   \path[thin] (a2) edge (a1);
%   \path[thin] (a9) edge (a4);
%   \path[thin] (a7) edge (a2);
% %   both history
%   \path[thin] (a14) edge (a13);
%   \path[thin] (a13) edge (a12);
%   \path[thin] (a12) edge (a9);
%   \path[thin] (a12) edge (a10);
%   
%   \path[thin] (a9) edge (a7);
%   \path[thin] (a10) edge (a7);
%   \end{tikzpicture}
%   \caption{Alice deals with border and Bloom Filter nodes}
% \end{subfigure}
%  \begin{subfigure}{0.2\textwidth}
%   \centering
%   \begin{tikzpicture}[->,>=stealth',shorten >=1pt,auto,node distance=3cm,
%   thick,main node/.style={circle,fill=blue!20,draw,font=\sffamily\Large\bfseries}]
%   
%   \foreach \place/\x in {{(0,0)/1}, {(0,1)/2},{(0,2)/3},{(0,3)/4}, {(1,4.5)/5}, {(2,1)/6}, {(1.5,2)/7},{(2.5,2)/8},{(1,3)/9},{(2,3)/10},{(3,3)/11},{(2,4)/12},{(2,5)/13},{(2,6)/14}}
%   \node[arn_n] (a\x) at \place {\x};
% %   Alice history
% %   \path[thin] (a14) edge (a5);
%   \path[thin] (a5) edge (a4);
%   \path[thin] (a4) edge (a3);
%   \path[thin] (a3) edge (a2);
%   \path[thin] (a2) edge (a1);
%   \path[thin] (a9) edge (a4);
%   \path[thin] (a7) edge (a2);
% %   both history
%   \path[thin] (a14) edge (a13);
%   \path[thin] (a13) edge (a12);
% %   \path[thin] (a12) edge (a9);
%   \path[thin] (a12) edge (a10);
%   
%   \path[thin] (a9) edge (a7);
%   \path[thin] (a10) edge (a7);
% %   Bob history
%   \path[thin] (a12) edge (a11);
%   \path[thin] (a11) edge (a8);
%   \path[thin] (a10) edge (a8);
%   \path[thin] (a7) edge (a6);
%   \path[thin] (a8) edge (a6);
%   \end{tikzpicture}
%   \caption{Bob notices 5 should have a parent}
%   \end{subfigure}%
% \end{figure}

